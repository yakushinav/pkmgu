\lecture{Лекция 10}{lec10}
\subtitle{Лекция 10 --- Программирование: Числовые алгоритмы}

\frame[plain]
{\titlepage}	% Титульный слайд


\begin{frame}
\frametitle{Программирование}

\begin{center}

\Huge
Список алгоритмов	
\end{center}
\end{frame}

	\begin{frame}
\frametitle{Список алгоритмов}
\begin{enumerate}
	\item Битовые операции
	\item Большее из двух, из трех
	\item Алгоритм Евклида
	\item Перевод в другую систему счисления
	\item Сумма и количество цифр числа
	\item Нахождение делителей числа
	\item Определение простоты числа
	\item Решение линейных и квадратных уравнений
	\item Табулирование функции
	\item Поиск максимума функции
\end{enumerate}
\end{frame}

	


	\begin{frame}
\frametitle{Битовые операции}

Переменную можно охарактеризовать шестеркой атрибутов:
\setlength{\columnsep}{0.4cm}
\begin{multicols}{2}
\begin{enumerate}
\item имя
\item адрес
\item значение
\item тип
\item время жизни
\item область видимости.
\end{enumerate}

\columnbreak
 \small
Область видимости - это часть текста программы, в которой может быть использован данный объект. Объект считается видимым в блоке или в исходном файле, если в этом блоке или файле известны имя и тип объекта. 
\end{multicols}

\end{frame}

\begin{frame}[fragile]
\frametitle{Пример кода}

\lstinputlisting[language=Delphi]{prg/ex01.pas}

\end{frame}

