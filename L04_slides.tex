\lecture{Лекция 4}{lec4}
\subtitle{Лекция 4 --- Алгебра логики. Часть 1}

\frame[plain]
{\titlepage}	% Титульный слайд


\begin{frame}
\frametitle{Алгебра логики}

\begin{center}

\Huge
Основные понятия
	
\end{center}


\end{frame}


\begin{frame}
\frametitle{Алгебра логики}

\textbf{Алгебра логики (алгебра высказываний)}~--- раздел математической
логики, в котором изучаются логические операции над высказываниями. 

\pause
Чаще
всего предполагается, что высказывания могут быть только истинными или
ложными, то есть используется так~называемая \emph{бинарная} или
\emph{двоичная} логика, в отличие от, например, троичной логики.


\end{frame}



\begin{frame}
\frametitle{Высказывания}

Высказывание в логике --- это повествовательное предложение, содержание которого может быть или истинным, или ложным.

\pause
Логические константы <<истина>> и <<ложь>> называют истинностными значениями высказывания.

\pause
Обозначения логических констант:\\
<<истина>> ---  И (Истина), T (true), 1 (единица);\\
<<ложь>> ---  Л (Ложь), F (False),  0 (ноль).


\end{frame}

\begin{frame}
\frametitle{Высказывания}

Высказывания принято обозначать заглавными буквами латинского алфавита, само содержание высказывания записывают в фигурных скобках.
\pause
\begin{block}{Пример высказывания}
 $$ A=\{ 3=2 \}$$
\end{block}
\pause
\begin{block}{Пример не высказывания}
 Сегодня идет дождь\\
 Волга впадает в Каспийское море
\end{block}

\end{frame}

\begin{frame}
\frametitle{Высказывания}

Высказывания бывают:
\begin{enumerate}
	\item простые (элементарные, атомарные);
	\item составные --- составленные комбинации из простых высказываний посредством логических связок. 
\end{enumerate}
\end{frame}


\begin{frame}
\frametitle{Логическая функция}

Логическая функция~--- это отображение вида $f: \{0, 1\}^n \rightarrow {0, 1}$, где $n$~---
число булевых переменных. В общем случае это скалярная функция векторного переменного.\pause

Логическая функция зависит от набора аргументов и принимает значения 0 или 1.\pause

Логическая функция это составное высказывание истинность компонентов которого меняется.

\end{frame}

\begin{frame}
\frametitle{Таблица истинности}

Таблица истинности --- таблица, описывающая логическую функцию. \pause

Таблица истинности --- таблица, которая показывает, какие значения примет логическая функция при всех возможных наборах значений простых выражений, входящих в него.\pause

Если имеется функция $f(x_1, x_2, \ldots, x_n)$, то вектор, представляющий эту функцию, будет единственным и иметь длину $2^n$.

\end{frame}

\begin{frame}
\frametitle{Таблица истинности}

В таблице, задающей булеву функцию от $n$ переменных, должно быть ровно $2^n$ строки~--- как раз
столько различных значений может принимать набор из $n$ переменных. \pause

Наборы значений переменных 
принято выписывать в \textit{лексикографическом порядке} (по возрастанию).

\end{frame}

\begin{frame}
\frametitle{Таблица истинности}
\framesubtitle{Правило заполнения}
В первом столбце: половина 0, половина 1, во втором половина 0, половина 1 в каждой половине и т.д.

\only<1>{
\begin{tabular}{|c|c|c|}
\hline 
x & y & z\tabularnewline
\hline 
\hline 
 &  & \tabularnewline
\hline 
 &  & \tabularnewline
\hline 
 &  & \tabularnewline
\hline 
 &  & \tabularnewline
\hline 
 &  & \tabularnewline
\hline 
 &  & \tabularnewline
\hline 
 &  & \tabularnewline
\hline 
 &  & \tabularnewline
\hline 
\end{tabular}
}
\only<2>{
\begin{tabular}{|c|c|c|}
\hline 
x & y & z\tabularnewline
\hline 
\hline 
0 &  & \tabularnewline
\hline 
0 &  & \tabularnewline
\hline 
0 &  & \tabularnewline
\hline 
0 &  & \tabularnewline
\hline 
 &  & \tabularnewline
\hline 
 &  & \tabularnewline
\hline 
 &  & \tabularnewline
\hline 
 &  & \tabularnewline
\hline 
\end{tabular}
}
\only<3>{
\begin{tabular}{|c|c|c|}
\hline 
x & y & z\tabularnewline
\hline 
\hline 
0 &  & \tabularnewline
\hline 
0 &  & \tabularnewline
\hline 
0 &  & \tabularnewline
\hline 
0 &  & \tabularnewline
\hline 
1 &  & \tabularnewline
\hline 
1 &  & \tabularnewline
\hline 
1 &  & \tabularnewline
\hline 
1 &  & \tabularnewline
\hline 
\end{tabular}
}
\only<4>{
\begin{tabular}{|c|c|c|}
\hline 
x & y & z\tabularnewline
\hline 
\hline 
0 & 0 & \tabularnewline
\hline 
0 & 0 & \tabularnewline
\hline 
0 & 1 & \tabularnewline
\hline 
0 & 1 & \tabularnewline
\hline 
1 &  & \tabularnewline
\hline 
1 &  & \tabularnewline
\hline 
1 &  & \tabularnewline
\hline 
1 &  & \tabularnewline
\hline 
\end{tabular}
}
\only<5>{
\begin{tabular}{|c|c|c|}
\hline 
x & y & z\tabularnewline
\hline 
\hline 
0 & 0 & \tabularnewline
\hline 
0 & 0 & \tabularnewline
\hline 
0 & 1 & \tabularnewline
\hline 
0 & 1 & \tabularnewline
\hline 
1 & 0 & \tabularnewline
\hline 
1 & 0 & \tabularnewline
\hline 
1 & 1 & \tabularnewline
\hline 
1 & 1 & \tabularnewline
\hline 
\end{tabular}
}
\only<6>{
\begin{tabular}{|c|c|c|}
\hline 
x & y & z\tabularnewline
\hline 
\hline 
0 & 0 & 0\tabularnewline
\hline 
0 & 0 & 1\tabularnewline
\hline 
0 & 1 & \tabularnewline
\hline 
0 & 1 & \tabularnewline
\hline 
1 & 0 & \tabularnewline
\hline 
1 & 0 & \tabularnewline
\hline 
1 & 1 & \tabularnewline
\hline 
1 & 1 & \tabularnewline
\hline 
\end{tabular}
}
\only<7>{
\begin{tabular}{|c|c|c|}
\hline 
x & y & z\tabularnewline
\hline 
\hline 
0 & 0 & 0\tabularnewline
\hline 
0 & 0 & 1\tabularnewline
\hline 
0 & 1 & 0\tabularnewline
\hline 
0 & 1 & 1\tabularnewline
\hline 
1 & 0 & 0\tabularnewline
\hline 
1 & 0 & 1\tabularnewline
\hline 
1 & 1 & 0\tabularnewline
\hline 
1 & 1 & 1\tabularnewline
\hline 
\end{tabular}
}


\end{frame}


\begin{frame}
\frametitle{Таблица истинности}

Пример для функции трех переменных: $f(x,y,z)$ (функция большинства)

\begin{tabular}{|c|c|c|c|}
\hline 
x & y & z & f(x,y,z)\tabularnewline
\hline 
0 & 0 & 0 & 0\tabularnewline
\hline 
0 & 0 & 1 & 0\tabularnewline
\hline 
0 & 1 & 0 & 0\tabularnewline
\hline 
0 & 1 & 1 & 1\tabularnewline
\hline 
1 & 0 & 0 & 0\tabularnewline
\hline 
1 & 0 & 1 & 1\tabularnewline
\hline 
1 & 1 & 0 & 1\tabularnewline
\hline 
1 & 1 & 1 & 1\tabularnewline
\hline 
\end{tabular}


\end{frame}




\begin{frame}


\frametitle{Алгебра логики}

\begin{center}

\Huge
Логические операции	
\end{center}


\end{frame}

\begin{frame}
\frametitle{Отрицание}

Отрицанием высказывания А называется новое высказывание, образованное из данного с помощью логической связки~«не~…»~(«неверно, что…»), и принимающее ложное значение, когда А истинно и наоборот.

Обозначение: $\overline{x}$ или $\neg x$

Таблица истинности

\begin{tabular}{|c|c|}
\hline 
x & $\overline{x}$\tabularnewline
\hline 
0 & 1\tabularnewline
\hline 
1 & 0\tabularnewline
\hline 
\end{tabular}

\end{frame}

\begin{frame}
\frametitle{Конъюнкция}

Конъюнкция (логиеское И) двух высказываний истинна тогда и только тогда, когда оба высказывания истинны.

Обозначение: $x\wedge y$ или $xy$, $x*y$, $x\&y$, $x\cap y$.

Таблица истинности

\begin{tabular}{|c|c|c|}
\hline
$A$ & $B$ & $A \wedge B$\\
\hline
\hline
\red{0} & \red{0} & \red{0}\\
\hline
\red{0} & \green{1} & \red{0}\\
\hline
\green{1} & \red{0} & \red{0}\\
\hline
\green{1} & \green{1} & \green{1}\\
\hline
\end{tabular}

\end{frame}

\begin{frame}
\frametitle{Дизъюнкция}

Дизъюнкиция (логиеское ИЛИ) двух высказываний истинна когда хотябы одно высказывание истинно истинны.

Обозначение: $x\vee y$ или $x+y$, $x||y$, $x\cup y$.

Таблица истинности

\begin{tabular}{|c|c|c|}
\hline
$A$ & $B$ & $A \vee B$\\
\hline
\hline
\red{0} & \red{0} & \red{0}\\
\hline
\red{0} & \green{1} & \red{1}\\
\hline
\green{1} & \red{0} & \red{1}\\
\hline
\green{1} & \green{1} & \green{1}\\
\hline
\end{tabular}

\end{frame}



\begin{frame}
\frametitle{Импликация}

Импликация (ЕСЛИ ... ТО, логический вывод) двух высказываний ложна, когда посылка истинна, а заключение ложно.

Обозначение: $x\rightarrow y$.

Таблица истинности

\begin{tabular}{|c|c|c|}
\hline
$A$ & $B$ & $A \rightarrow B$\\
\hline
\hline
\red{0} & \red{0} & \red{1}\\
\hline
\red{0} & \green{1} & \red{1}\\
\hline
\green{1} & \red{0} & \red{0}\\
\hline
\green{1} & \green{1} & \green{1}\\
\hline
\end{tabular}

\end{frame}

\begin{frame}
\frametitle{Тождество}

Тождество (эквивалентность, А тоже самое что В) двух высказываний истинна, когда оба высказывания принимают одинаковые значения истинности.

Обозначение: $x\equiv y$ или $x \sim y$.

Таблица истинности

\begin{tabular}{|c|c|c|}
\hline
$A$ & $B$ & $A \equiv B$\\
\hline
\hline
\red{0} & \red{0} & \red{1}\\
\hline
\red{0} & \green{1} & \red{0}\\
\hline
\green{1} & \red{0} & \red{0}\\
\hline
\green{1} & \green{1} & \green{1}\\
\hline
\end{tabular}

\end{frame}

\begin{frame}
\frametitle{Исключающее ИЛИ}

Исключающее ИЛИ (анти эквивалентность, А не тоже самое что В) двух высказываний истинна, когда оба высказывания принимают разные значения истинности.

Обозначение: $x \oplus y$.

Таблица истинности

\begin{tabular}{|c|c|c|}
\hline
$A$ & $B$ & $A \oplus B$\\
\hline
\hline
\red{0} & \red{0} & \red{0}\\
\hline
\red{0} & \green{1} & \red{1}\\
\hline
\green{1} & \red{0} & \red{1}\\
\hline
\green{1} & \green{1} & \green{0}\\
\hline
\end{tabular}

\end{frame}

\begin{frame}
\frametitle{Логические функции: перечисление}

Важное свойство булевых функции~--- их число конечно для заданного $n$ и равно
$2^{2^n}$. То есть, каждая функция может быть задана своей \emph{таблицей истинности} или просто
\emph{пронумерована}.  

При $n = 0$ существует только две функции-константы: 0 и 1.

Рассмотрим все функции для $n = 1$:

\begin{center}
  \begin{tabular}{|c|c|c|c|c|}
    \hline
    $x$ & $f_1$ & $f_2$ & $f_3$ & $f_4$\\ \hline
    0 & 0 & 0 & 1 & 1\\ \hline
    1 & 0 & 1 & 0 & 1\\ \hline
  \end{tabular}
\end{center}

$f_1(x) = 0, f_4(x) = 1$~--- константы, $f_2(x) = x$~--- тождественная функция, $f_3(x) =
\overline{x}$~--- отрицание.
\end{frame}

\begin{frame}
\frametitle{Логические функции: перечисление}
Рассмотрим все функции для $n = 2$: \\ \scriptsize
  \begin{tabular}{|c|c|c|c|c|c|c|c|c|c|c|c|c|c|c|c|c|c|}
    \hline
    x & y & 1 & 2 & 3 & 4 & 5 & 6 & 7 & 8 & 9 & 10 & 11 & 12 & 13 & 14 & 15 & 16\\ \hline
    0 & 0 & 0 & 0 & 0 & 0 & 0 & 0 & 0 & 0 & 1 & 1 & 1 & 1 & 1 & 1 & 1 & 1\\ \hline
    0 & 1 & 0 & 0 & 0 & 0 & 1 & 1 & 1 & 1 & 0 & 0 & 0 & 0 & 1 & 1 & 1 & 1\\ \hline
    1 & 0 & 0 & 0 & 1 & 1 & 0 & 0 & 1 & 1 & 0 & 0 & 1 & 1 & 0 & 0 & 1 & 1\\ \hline
    1 & 1 & 0 & 1 & 0 & 1 & 0 & 1 & 0 & 1 & 0 & 1 & 0 & 1 & 0 & 1 & 0 & 1\\ \hline
  \end{tabular}
\normalsize
$f_1(x, y) = 0, f_{16}(x, y) = 1$~--- константы, \\ \pause
$f_2(x, y) =x~\wedge~y$, $f_{15}(x, y) =\overline{x~\wedge~y}$(штрих Шеффера)  \\\pause
$f_8(x, y) = x~\vee~y$, $f_9(x, y) = \overline{x~\vee~y}$(стрелка Пирса)\\\pause
$f_7(x, y) = x~\oplus~y$, $f_{10}(x, y) = x~\equiv~y$\\\pause
$f_{14}(x, y) =x~\rightarrow~y$,$f_{3}(x, y) =\overline{x~\rightarrow~y}$  \\ \pause
$f_{12}(x, y) =y~\rightarrow~x$,$f_{5}(x, y) =\overline{y~\rightarrow~x}$  \\ \pause

Таким образом, каждая булева функция может быть представлена как в виде таблицы истинности,
так и в виде формулы. 

\end{frame}

\begin{frame}
\frametitle{Catch phrase логических операций}


\begin{center}
\begin{tabular}{|c|c|}
\hline 
Операция & Catch phrase \tabularnewline
\hline 
\hline 
$\overline{\textcolor[rgb]{1,1,1}{x}}$ & НЕ A \pause \tabularnewline 
\hline 
$\wedge$ & A И B \pause \tabularnewline 
\hline 
$\vee$ & A ИЛИ B  \pause \tabularnewline 
\hline 
$\oplus$ & ИЛИ A ИЛИ В \pause \tabularnewline 
\hline 
$\rightarrow$ & ЕСЛИ А ТО B, А ВЛЕЧЕТ В  \pause \tabularnewline
\hline 
$\equiv$ & А ТОГДА И ТОЛЬКО ТОГДА КОГДА B \tabularnewline  
\hline 
\end{tabular}\end{center}
\end{frame}

\begin{frame}
\frametitle{Приоритет логических операций}

Сначала выполняется операция с самым низким приоритетом.
\begin{center}
\begin{tabular}{|c|c|c|}
\hline 
Приоритет & Операция & Описание\tabularnewline
\hline 
0 & $\overline{x}$ & Отрицание \tabularnewline
\hline 
1 & $\wedge$ & Конъюнкция\tabularnewline
\hline 
2 & $\vee,\oplus$ & Дизъюнкция, Исключающее ИЛИ\tabularnewline
\hline 
3 & $\rightarrow$ & Импликация \tabularnewline
\hline 
4 & $\equiv$ & Тождество\tabularnewline
\hline 
\end{tabular}
\end{center}
\end{frame}

\begin{frame}
\frametitle{Приоритет логических операций}
\framesubtitle{Пример}

\only<1>{$$ \overline{x} \rightarrow (x \vee \overline{y} \wedge z) $$}
\only<2>{ $$ \underbrace{\overline{x}}_{1} \rightarrow (x \vee \underbrace{\overline{y}}_{2} \wedge z) $$}
\only<3>{ $$ \underbrace{\overline{x}}_{1} \rightarrow (x \vee \underbrace{\underbrace{\overline{y}}_{2} \wedge z}_{3}) $$}
\only<4>{ $$ \underbrace{\overline{x}}_{1} \rightarrow (\underbrace{x \vee \underbrace{\underbrace{\overline{y}}_{2} \wedge z}_{3}}_{4}) $$}
\only<5>{ $$ \underbrace{\underbrace{\overline{x}}_{1} \rightarrow (\underbrace{x \vee \underbrace{\underbrace{\overline{y}}_{2} \wedge z}_{3}}_{4})}_{5} $$}

\end{frame}



\begin{frame}[fragile,squeeze]{Таблица истинности}
Рассмотрим формулу $F(x_1,x_2,x_3) = {\bar x}_1 \wedge (x_2\oplus x_3)$ 
Составим таблицу истинности~--- сначала выпишем столбец ${\bar x}_1$, затем столбец $(x_2\oplus x_3)$,
а затем конъюнкцию этих двух столбцов.

\medskip
\begin{tabular}{|c|c|c||c|c|c|}
  \hline
  $x_1$ & $x_2$ & $x_3$ & $\bar x_1$ & $x_2\oplus x_3$ & $F$ \\
  \hline
   0  &  0  & 0         & 1          &  0              & 0 \\
   0  &  0  & 1         & 1          &  1              & 1 \\
   0  &  1  & 0         & 1          &  1              & 1 \\
   0  &  1  & 1         & 1          &  0              & 0 \\
   1  &  0  & 0         & 0          &  0              & 0 \\
   1  &  0  & 1         & 0          &  1              & 0 \\
   1  &  1  & 0         & 0          &  1              & 0 \\
   1  &  1  & 1         & 0          &  0              & 0 \\
  \hline
\end{tabular}
\end{frame}

\begin{frame}
\frametitle{Алгебра логики}
\begin{center}

\Huge
Законы (формулы) алгебры логики
\end{center}


\end{frame}

\begin{frame}
\frametitle{Основные законы алгебры логики}
Закон двойного отрицания: $ \overline{\overline{x}}=x $  \pause

Коммутативность:\\
 $ A \vee B = B \vee A$\\
 $ A \wedge B = B \wedge A$\\ \pause

Ассоциативность:\\
 $ A \vee (B \vee C) = (A \vee B) \vee C$\\
 $ A \wedge (B \wedge C) = (A \wedge B) \wedge C$\\ \pause

Дистибутивность:\\
$ A \wedge (B \vee C) = (A \wedge B) \vee (A\wedge C)$\\ 
$ A \vee (B \wedge C) = (A \vee B) \wedge (A \vee C)$\\
 

\end{frame}

\begin{frame}
\frametitle{Основные законы алгебры логики}
Законы де Моргана:
$$ \overline{A \vee B} = \overline{A} \wedge \overline{B} $$
$$ \overline{A \wedge B} = \overline{A} \vee \overline{B} $$ \pause

Поглощение:
$$ A \vee A\wedge B =A$$ \pause

Склейка
$$ x_1\wedge x_2 \wedge x_3 \vee x_1\wedge \overline{x_2} \wedge x_3 = x_1 \wedge x_3$$

\end{frame}

\begin{frame}
\frametitle{Примитивные формулы}
Конъюнкция:\\
$ x \wedge x$ =\pause $x$ (идемпотентность)\\
$ x \wedge \overline{x}$ =\pause $0$\\
$ x \wedge 1$ =\pause $x$\\
$ x \wedge 0$ =\pause $0$

Дизъюнкция:\\
$ x \vee x$ =\pause $x$\\
$ x \vee \overline{x}$ =\pause $1$ (закон исключения третьего)\\
$ x \vee 1$ =\pause $1$\\
$ x \vee 0$ =\pause $x$



\end{frame}

\begin{frame}
\frametitle{Примитивные формулы}
Тождество:\\
$ x \equiv x$ =\pause $1$ \\
$ x \equiv \overline{x}$ =\pause $0$\\
$ x \equiv 1$ =\pause $x$\\
$ x \equiv 0$ =\pause $\overline{x}$

Исключающее ИЛИ:\\
$ x \oplus x$ =\pause $0$\\
$ x \oplus \overline{x}$ =\pause $1$\\
$ x \oplus 1$ =\pause $\overline{x}$\\
$ x \oplus 0$ =\pause $x$



\end{frame}

\begin{frame}
\frametitle{Примитивные формулы}
Импликация:\\
$ x \rightarrow x$ =\pause $1$ \\
$ x \rightarrow \overline{x}$ =\pause $\overline{x}$\\
$ \overline{x} \rightarrow x $ =\pause $x$\\
$ x \rightarrow 1$ =\pause $1$\\
$ 1 \rightarrow x$ =\pause $x$\\
$ x \rightarrow 0$ =\pause $\overline{x}$
$ 0 \rightarrow x$ =\pause $1$

\end{frame}

\begin{frame}
\frametitle{Представление формул в стандартном базисе}
Набор операций: $\overline{\textcolor[rgb]{1,1,1}{x}}, \wedge, \vee$ составляет базис. \pause
 
Через базис можно выразить любые логические функции.

Например:\\
$x \rightarrow y = \overline{x} \vee y$\\ \pause
$x \oplus y = \overline{x}\wedge y \vee x\wedge \overline{y}$\\ \pause
$x \equiv y = \overline{x}\wedge \overline{y} \vee x\wedge y$\\ \pause



\end{frame}