\lecture{Лекция 10}{lec10}
\subtitle{Лекция 10 --- Программирование: Числовые алгоритмы}

\frame[plain]
{\titlepage}	% Титульный слайд


\begin{frame}
\frametitle{Программирование}

\begin{center}

\Huge
Список алгоритмов	
\end{center}
\end{frame}

	\begin{frame}
\frametitle{Список алгоритмов}
\begin{enumerate}
	\item Битовые операции
	\item Большее из двух, из трех
	\item Алгоритм Евклида
	\item Сумма цифр числа
	\item Перевод в другую систему счисления
	\item Нахождение делителей числа
	\item Определение простоты числа
	\item Решение линейных и квадратных уравнений
	\item Табулирование функции
	\item Поиск максимума функции
\end{enumerate}
\end{frame}

	


	\begin{frame}
\frametitle{Битовые операции}

Над битами двух целых операндов можно выполнять логические операции: \texttt{not, and, or, xor}. 

Отличие между побитовыми и логическими операциями состоит в том, что побитовые (поразрядные) операции выполняются над отдельными битами операндов, а не над их значением в десятичном (обычно) представлении.

\end{frame}

	\begin{frame}
\frametitle{Битовые операции}

В Паскаль определены еще две операции shl и shr, которые сдвигают последовательность битов на заданное число позиций влево или вправо соответственно. 

При этом биты, которые выходят за разрядную сетку, теряются. 

При выполнении операции shl освободившиеся справа биты заполняются нулями. 

При выполнении операции shr освободившиеся слева биты заполняются единицами при сдвиге вправо отрицательных значений и нулями в случае положительных значений.

\end{frame}

\begin{frame}[fragile]
\frametitle{Пример битовых операций}

\lstinputlisting[style=CStyle]{prg/ex00.pas}

\end{frame}

\begin{frame}[fragile]
\frametitle{Задачи на битовые операции}

\begin{itemize}
 \item Определить является ли число степенью двойки
 \item Проверить число на четность
 \item вычислить $(-1)^n$
 \item Обнулить крайний правый бит числа
 \item Получить следующее большее число с тем же количеством единичных бита
 \item и другие (Г. Уоррен Алгоритмические трюки для программистов)
 
\end{itemize}


\end{frame}



\begin{frame}[fragile]
\frametitle{Большее из двух}

\lstinputlisting[style=CStyle]{prg/ex01.pas}

При всей простоте это довольно популярный алгоритм. Сравнение двух переменных может быть реализовано достаточно сложным способом.

Например, попробуйте придумать сравнение при котором $7>65$.

\end{frame}

\begin{frame}[fragile]
\frametitle{Большее из трех}

\lstinputlisting[style=CStyle]{prg/ex02.pas}

Первый вариант хуже чем второй и третий, так как в некоторых случаях выполнится несколько условных операторов.

\end{frame}

\begin{frame}[fragile]
\frametitle{Алгоритм Евклида}

Алгоритм Евклида — это способ нахождения наибольшего общего делителя (НОД) двух целых чисел. Оригинальная версия алгоритма, когда НОД находится вычитанием, была открыта Евклидом (III в. до н. э). В настоящее время чаще при вычислении НОД алгоритмом Евклида используют деление, так как данный метод эффективнее.

Наибольший общий делитель пары чисел – это самое большое число, которое нацело делит оба числа пары. 

\end{frame}

\begin{frame}[fragile]
\frametitle{Алгоритм Евклида}
\framesubtitle{Вычитание}

\lstinputlisting[style=CStyle]{prg/ex03.pas}

\end{frame}

\begin{frame}[fragile]
\frametitle{Алгоритм Евклида}
\framesubtitle{Деление}

\lstinputlisting[style=CStyle]{prg/ex04.pas}

\end{frame}

\begin{frame}[fragile]
\frametitle{Алгоритм Евклида}
\framesubtitle{Рекурсия}

\lstinputlisting[style=CStyle]{prg/ex05.pas}

\end{frame}

\begin{frame}[fragile]
\frametitle{Сумма цифр числа}

Алгоритм решения задачи: 
\begin{enumerate}
\item сумме присвоить ноль.
\item Пока n больше нуля
\item найти остаток от деления n на 10 (т.е. последнюю цифру числа), добавить его к сумме и увеличить произведение;
\item избавиться от последнего разряда числа n путем деления нацело на число 10.
\end{enumerate}

    
\end{frame}

\begin{frame}[fragile]
\frametitle{Сумма цифр числа}

\lstinputlisting[style=CStyle]{prg/ex06.pas}

Как можно модифицировать алгоритм, чтобы найти количество цифр? наибольшую цифру?

Что произойдет если 10 заменить на 8?

    
\end{frame}

\begin{frame}[fragile]
\frametitle{Перевод в другую систему счисления}

Используем стандартный алгоритм: деление с остатком на основание системы счисления. Последовательность остатков образует цифры нового числа. 

Основная проблема: цифры нового числа получаются в обратном порядке.

Способы решения:
\begin{enumerate}
 \item рекурсивная функция
 \item запись в новое число
 \item массив
\end{enumerate}
    
\end{frame}

\begin{frame}[fragile]
\frametitle{Перевод из десятичной в двоичную}
\framesubtitle{Рекурсия}

\lstinputlisting[style=CStyle]{prg/ex07.pas}

    
\end{frame}

\begin{frame}[fragile]
\frametitle{Перевод из десятичной в двоичную}
\framesubtitle{Запись в новое число}

\lstinputlisting[style=CStyle]{prg/ex08.pas}

    
\end{frame}


\begin{frame}[fragile]
\frametitle{Делители числа}
\framesubtitle{Вариант 1}
\lstinputlisting[style=CStyle]{prg/ex09.pas}

    
\end{frame}

\begin{frame}[fragile]
\frametitle{Делители числа}
\framesubtitle{Вариант 2}
\lstinputlisting[style=CStyle]{prg/ex10.pas}

    
\end{frame}

\begin{frame}[fragile]
\frametitle{Простое число}
Используем тот факт, что если у числа $n$ есть делитель, то он находится в интервале $[2;\sqrt{n}]$

\lstinputlisting[style=CStyle]{prg/ex11.pas}
    
\end{frame}

\begin{frame}[fragile]
\frametitle{Простое число}

\begin{enumerate}
 \item использовать меньший интервал (гипотеза Римана)
 \item использовать список простых числел в качестве делителей
 \item решето Эратосфена
\end{enumerate}

    
\end{frame}

\begin{frame}[fragile]
\frametitle{Решение линейного уравнения}

Решим уравнение $ax+b=0$.

\lstinputlisting[style=CStyle]{prg/ex12.pas}
    
\end{frame}

\begin{frame}[fragile]
\frametitle{Решение квадратного уравнения}

Решим уравнение $ax^2+bx+c=0$.

\lstinputlisting[style=CStyle]{prg/ex13.pas}
    
\end{frame}

\begin{frame}[fragile]
\frametitle{Табулирование функции}

Задача табулирования (вывод значений с шагом) функции $f(x)=x^2-3x+2$ на отрезке $[a,b]$ с шагом $h$.

\lstinputlisting[style=CStyle]{prg/ex14.pas}
    
\end{frame}

\begin{frame}[fragile]
\frametitle{Наибольшее заначение функции на отезке с шагом}

Задача поиск максимума функции $f(x)=x^2-3x+2$ на отрезке $[a,b]$ с шагом $h$.
\lstinputlisting[style=CStyle]{prg/ex15.pas}
    
\end{frame}

\begin{frame}
\frametitle{Программирование}

\begin{center}

\Huge
Примеры решения задач
\end{center}
\end{frame}

\begin{frame}[fragile]
\frametitle{Пример 1}

Запишите число, которое будет напечатано в результате выполнения программы. Решение оформляем в таблице.
\begin{multicols}{2}
\begin{lstlisting}[style=CStyle]
var s, n: integer;
begin
  s := 33;
  n := 1;
  while s > 0 do begin
    s := s – 7;
    n := n * 3
  end;
  writeln(n)
end.
\end{lstlisting}
\columnbreak
\begin{tabular}{|c|c|c|}
\hline 
s & n & s>0\tabularnewline
\hline 
\hline 
33 & 1 & 1\tabularnewline
\hline 
26 & 3 & 1\tabularnewline
\hline 
19 & 9 & 1\tabularnewline
\hline 
12 & 27 & 1\tabularnewline
\hline 
5 & 81 & 1\tabularnewline
\hline 
-2 & 243 & 0\tabularnewline
\hline 
\end{tabular}

Ответ: 243.
\end{multicols}


    
\end{frame}

\begin{frame}[fragile]
\frametitle{Пример 2}

При каком наибольшем введенном числе d после выполнения программы будет напечатано 55?
\begin{multicols}{2}
\begin{lstlisting}[style=CStyle]
var n, s, d: integer;
begin
  readln(d);
  n := 0;
  s := 0;
  while s <= 365 do begin
    s := s + d;
    n := n + 5
  end;
  write(n)
end.
\end{lstlisting}
\columnbreak
\small
Мы прибавляем 5 к переменной $n$. Сколько раз нужно прибавить, чтобы получить 55? - 11.
Следовательно цикл должен выполнится 11 раз.

$s=k\cdot d$. Для $k=10$ цикл выполняется, а для $k=11$ не выполняется.

$10d\leq 365$, $11d>365$, $(d\leq \frac{365}{10}=36.5)and(d>\frac{365}{11}=33.18)$

$d\in[34;36]$, Ответ: 36
\end{multicols}


    
\end{frame}

\begin{frame}[fragile]
\frametitle{Пример 3}
Ниже записан алгоритм. Укажите наименьшее пятизначное число  , при вводе которого алгоритм печатает сначала 4, а потом 2.
\begin{lstlisting}[style=CStyle]
var x, y, a, b: longint;
begin
  a := 0;
  b := 0;
  readln(x);
  while x > 0 do begin
    y := x mod 10;
    if y > 3 then a := a + 1;
    if y < 8 then b := b + 1;
    x := x div 10
  end;
  writeln(a);
  writeln(b)
end.
\end{lstlisting}
\end{frame}

\begin{frame}[fragile]
\frametitle{Пример 3}
Ниже записан алгоритм. Укажите наименьшее пятизначное число  , при вводе которого алгоритм печатает сначала 4, а потом 2.

Шаг 1. Нужно определить, что делает алгоритм. Для этого возьмем некоторое число и прогоним алгоритм.
Число следует выбрать <<правильно>>. Например хорошее число $x=4237$. Почему?
\begin{multicols}{2}
\begin{tabular}{|c|c|c|c|}
\hline 
x & a & b & y\tabularnewline
\hline 
\hline 
4237 & 0 & 0 & \tabularnewline
\hline 
423 & 1 & 1 & 7\tabularnewline
\hline 
42 & 1 & 2 & 3\tabularnewline
\hline 
4 & 1 & 3 & 2\tabularnewline
\hline 
0 & 2 & 4 & 4\tabularnewline
\hline 
\end{tabular}
\columnbreak

a --- количество цифр числа, больших 3.\\
b --- количество цифр числа, меньших 8.

\end{multicols}



\end{frame}

\begin{frame}[fragile]
\frametitle{Пример 3}
Ниже записан алгоритм. Укажите наименьшее пятизначное число  , при вводе которого алгоритм печатает сначала 4, а потом 2.

Шаг 2. Нужно найти наименьшее пятизначное число у которого 4 цифры больше 3 и 2 цифры меньше 8.

Так как цифр 5, то как минимум одна совпадает.

$x_{min}=14888$
\end{frame}


\begin{frame}[fragile]
\frametitle{Пример 4}
Получив на вход число x, эта программа печатает два числа, a и b. Укажите наибольшее из чисел x, при вводе которых алгоритм печатает сначала 2, а потом 8.
\begin{lstlisting}[style=CStyle]
var x, a, b: longint;
begin
  readln(x);
  a:=0; b:=0;
  while x > 0 do begin
    a:= a + 1;
    b:= b + (x mod 100);
    x:= x div 100;
  end;
  writeln(a); write(b);
end.
\end{lstlisting}
\pause Ответ: 800
\end{frame}

\begin{frame}[fragile]
\frametitle{Пример 5}
Ниже записан алгоритм. Сколько существует таких чисел x, при вводе которых алгоритм печатает сначала 2, а потом 12?
\begin{lstlisting}[style=CStyle]
var x, a, b: longint;
begin
  readln(x);
  a := 0; b := 0;
  while x>0 do begin
    a := a + 1;
    b := b + (x mod 100);
    x := x div 100
  end;
  writeln(a); write(b)
end.
\end{lstlisting}
\pause
Ответ: 12
\end{frame}

\begin{frame}[fragile]
\frametitle{Пример 6}
Ниже приведён алгоритм. Укажите наименьшее из таких чисел x, большее, чем 200, при вводе которого алгоритм напечатает 50.
\begin{lstlisting}[style=CStyle]
var x, L, M: longint;
begin
  readln(x);
  L := 2*x-20;
  M := 2*x+30;
  while L <> M do begin
    if L > M then
      L := L - M
    else
      M := M - L;
  end;
  writeln(M);
end.
\end{lstlisting}
\pause
Ответ: 210
\end{frame}

\begin{frame}[fragile]
\frametitle{Пример 7}
Определите, какое число будет напечатано:
\small
\begin{lstlisting}[style=CStyle]
var a,b,t,M,R:integer;
Function F(x: integer):integer;
begin
  F := 5*(x+10)*(x+2)+2;
end;
BEGIN
  a := -20; b := 20; M := a; R:= F(a);
  for t := a to b do
    if (F(t)<R) then begin
      M := t;
      R:= F(t);
    end;
  write(M);
END.
\end{lstlisting}\pause
Ответ: 30
\end{frame}

\begin{frame}[fragile]
\frametitle{Пример 8}
Определите, количество чисел K, для которых следующая программа выведет такой же результат, что и для K = 18:
\begin{lstlisting}[style=CStyle]
var i, k: integer;
function F(x:integer):integer;
begin
  F:=x*x;
end;
begin
  i := 0;
  readln(K);
  while F(i) < K do 
    i:=i+1;
  writeln(i);  
end.
\end{lstlisting}
\pause
Ответ: 9
\end{frame}

\begin{frame}[fragile]
\frametitle{Пример 9}
Напишите в ответе наименьшее значение входной переменной k, при котором программа выдаёт тот же ответ, что и при входном значении k = 14. 
\small
\begin{lstlisting}[style=CStyle]
var k, i : longint;
function f(n: longint): longint;
begin f := n * n * n; end;
function g(n: longint): longint;
begin g := 5*n + 1; end;
begin
  readln(k);
  i := 1;
  while f(i) < g(k) do
    i := i+1;
  writeln(i)
end.
\end{lstlisting}
\pause
Ответ: 13
\end{frame}

\begin{frame}[fragile]
\frametitle{Пример 10}
Напишите в ответе количество различных значений входной переменной a из интервала от 1 до 100 (включая границы), при которых программа выдаёт тот же ответ, что и при входном значении a = 20. Значение a = 20 также включается в подсчёт различных значений a.
\begin{lstlisting}[style=CStyle]
var  i, k, a: integer;
function f(x: integer; y: integer): integer;
begin
  if x = y then 
    f := x else 
  if x > y then f := f(x - y, y)
  else f := f(x, y - x);
end;

\end{lstlisting}

\end{frame}

\begin{frame}[fragile]
\frametitle{Пример 10}
Напишите в ответе количество различных значений входной переменной a из интервала от 1 до 100 (включая границы), при которых программа выдаёт тот же ответ, что и при входном значении a = 20. Значение a = 20 также включается в подсчёт различных значений a.
\begin{lstlisting}[style=CStyle]
var  i, k, a: integer;
begin
  k := 0;
  readln(a);  
  for i := 1 to a do   
    if f(i, 4) = 2 then k := k + 1;
  writeln(k); 
end.
\end{lstlisting}
\pause
Ответ: 4
\end{frame}


\begin{frame}[fragile]
\frametitle{Пример 11}
Какое число будет напечатано:
\begin{lstlisting}[style=CStyle]
var i, k: integer;
function f(x: integer): integer;
begin
  if x > 0 then f := x mod 10 + f(x div 10) 
					 else f := 0;
end;
begin
 k := 0;  
 for i := 1000 to 9999 do
  if f(i mod 10) = 1 then
   if f(i div 100) = f(i mod 100) then k := k + 1;
  writeln(k);
end.
\end{lstlisting}
\pause
Ответ: 54
\end{frame}