\lecture{Лекция 10}{lec10}
\subtitle{Лекция 10 --- Программирование: Числовые алгоритмы}

\frame[plain]
{\titlepage}	% Титульный слайд


\begin{frame}
\frametitle{Программирование}

\begin{center}

\Huge
Список алгоритмов	
\end{center}
\end{frame}

	\begin{frame}
\frametitle{Список алгоритмов}
\begin{enumerate}
	\item Битовые операции
	\item Большее из двух, из трех
	\item Алгоритм Евклида
	\item Сумма цифр числа
	\item Перевод в другую систему счисления
	\item Нахождение делителей числа
	\item Определение простоты числа
	\item Решение линейных и квадратных уравнений
	\item Табулирование функции
	\item Поиск максимума функции
\end{enumerate}
\end{frame}

	


	\begin{frame}
\frametitle{Битовые операции}

Над битами двух целых операндов можно выполнять логические операции: \texttt{not, and, or, xor}. 

Отличие между побитовыми и логическими операциями состоит в том, что побитовые (поразрядные) операции выполняются над отдельными битами операндов, а не над их значением в десятичном (обычно) представлении.

\end{frame}

	\begin{frame}
\frametitle{Битовые операции}

В Паскаль определены еще две операции shl и shr, которые сдвигают последовательность битов на заданное число позиций влево или вправо соответственно. 

При этом биты, которые выходят за разрядную сетку, теряются. 

При выполнении операции shl освободившиеся справа биты заполняются нулями. 

При выполнении операции shr освободившиеся слева биты заполняются единицами при сдвиге вправо отрицательных значений и нулями в случае положительных значений.

\end{frame}

\begin{frame}[fragile]
\frametitle{Пример битовых операций}

\lstinputlisting[style=CStyle]{prg/ex00.pas}

\end{frame}

\begin{frame}[fragile]
\frametitle{Задачи на битовые операции}

\begin{itemize}
 \item Определить является ли число степенью двойки
 \item Проверить число на четность
 \item вычислить $(-1)^n$
 \item Обнулить крайний правый бит числа
 \item Получить следующее большее число с тем же количеством единичных бита
 \item и другие (Г. Уоррен Алгоритмические трюки для программистов)
 
\end{itemize}


\end{frame}



\begin{frame}[fragile]
\frametitle{Большее из двух}

\lstinputlisting[style=CStyle]{prg/ex01.pas}

При всей простоте это довольно популярный алгоритм. Сравнение двух переменных может быть реализовано достаточно сложным способом.

Например, попробуйте придумать сравнение при котором $7>65$.

\end{frame}

\begin{frame}[fragile]
\frametitle{Большее из трех}

\lstinputlisting[style=CStyle]{prg/ex02.pas}

Первый вариант хуже чем второй и третий, так как в некоторых случаях выполнится несколько условных операторов.

\end{frame}

\begin{frame}[fragile]
\frametitle{Алгоритм Евклида}

Алгоритм Евклида — это способ нахождения наибольшего общего делителя (НОД) двух целых чисел. Оригинальная версия алгоритма, когда НОД находится вычитанием, была открыта Евклидом (III в. до н. э). В настоящее время чаще при вычислении НОД алгоритмом Евклида используют деление, так как данный метод эффективнее.

Наибольший общий делитель пары чисел – это самое большое число, которое нацело делит оба числа пары. 

\end{frame}

\begin{frame}[fragile]
\frametitle{Алгоритм Евклида}
\framesubtitle{Вычитание}

\lstinputlisting[style=CStyle]{prg/ex03.pas}

\end{frame}

\begin{frame}[fragile]
\frametitle{Алгоритм Евклида}
\framesubtitle{Деление}

\lstinputlisting[style=CStyle]{prg/ex04.pas}

\end{frame}

\begin{frame}[fragile]
\frametitle{Алгоритм Евклида}
\framesubtitle{Рекурсия}

\lstinputlisting[style=CStyle]{prg/ex05.pas}

\end{frame}

\begin{frame}[fragile]
\frametitle{Сумма цифр числа}

Алгоритм решения задачи: 
\begin{enumerate}
\item сумме присвоить ноль.
\item Пока n больше нуля
\item найти остаток от деления n на 10 (т.е. последнюю цифру числа), добавить его к сумме и увеличить произведение;
\item избавиться от последнего разряда числа n путем деления нацело на число 10.
\end{enumerate}

    
\end{frame}

\begin{frame}[fragile]
\frametitle{Сумма цифр числа}

\lstinputlisting[style=CStyle]{prg/ex06.pas}

Как можно модифицировать алгоритм, чтобы найти количество цифр? наибольшую цифру?

Что произойдет если 10 заменить на 8?

    
\end{frame}

\begin{frame}[fragile]
\frametitle{Перевод в другую систему счисления}

Используем стандартный алгоритм: деление с остатком на основание системы счисления. Последовательность остатков образует цифры нового числа. 

Основная проблема: цифры нового числа получаются в обратном порядке.

Способы решения:
\begin{enumerate}
 \item рекурсивная функция
 \item запись в новое число
 \item массив
\end{enumerate}
    
\end{frame}

\begin{frame}[fragile]
\frametitle{Перевод из десятичной в двоичную}
\framesubtitle{Рекурсия}

\lstinputlisting[style=CStyle]{prg/ex07.pas}

    
\end{frame}

\begin{frame}[fragile]
\frametitle{Перевод из десятичной в двоичную}
\framesubtitle{Запись в новое число}

\lstinputlisting[style=CStyle]{prg/ex08.pas}

    
\end{frame}


\begin{frame}[fragile]
\frametitle{Делители числа}
\framesubtitle{Вариант 1}
\lstinputlisting[style=CStyle]{prg/ex09.pas}

    
\end{frame}

\begin{frame}[fragile]
\frametitle{Делители числа}
\framesubtitle{Вариант 2}
\lstinputlisting[style=CStyle]{prg/ex10.pas}

    
\end{frame}

\begin{frame}[fragile]
\frametitle{Простое число}
Используем тот факт, что если у числа $n$ есть делитель, то он находится в интервале $[2;\sqrt{n}]$

\lstinputlisting[style=CStyle]{prg/ex11.pas}
    
\end{frame}

\begin{frame}[fragile]
\frametitle{Простое число}

\begin{enumerate}
 \item использовать меньший интервал (гипотеза Римана)
 \item использовать список простых числел в качестве делителей
 \item решето Эратосфена
\end{enumerate}

    
\end{frame}

\begin{frame}[fragile]
\frametitle{Решение линейного уравнения}

Решим уравнение $ax+b=0$.

\lstinputlisting[style=CStyle]{prg/ex12.pas}
    
\end{frame}

\begin{frame}[fragile]
\frametitle{Решение квадратного уравнения}

Решим уравнение $ax^2+bx+c=0$.

\lstinputlisting[style=CStyle]{prg/ex13.pas}
    
\end{frame}

\begin{frame}[fragile]
\frametitle{Табулирование функции}

Задача табулирования (вывод значений с шагом) функции $f(x)=x^2-3x+2$ на отрезке $[a,b]$ с шагом $h$.

\lstinputlisting[style=CStyle]{prg/ex14.pas}
    
\end{frame}

\begin{frame}[fragile]
\frametitle{Наибольшее заначение функции на отезке с шагом}

Задача поиск максимума функции $f(x)=x^2-3x+2$ на отрезке $[a,b]$ с шагом $h$.
\lstinputlisting[style=CStyle]{prg/ex15.pas}
    
\end{frame}

\begin{frame}
\frametitle{Программирование}

\begin{center}

\Huge
Примеры решения задач
\end{center}
\end{frame}
